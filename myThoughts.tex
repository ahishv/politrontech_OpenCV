\documentclass{article}
\usepackage[utf8]{inputenc}
\usepackage[english]{babel}
\usepackage{xcolor}
\usepackage{hyperref}   
\usepackage{tabulary} 
\begin{document}
\title{Initial thoughts on the problem of analysing truck drivers visually for various violations and events}
\author{Ahish.V}
\maketitle
\textbf{PROBLEM:} \\
    The problem Definition as I understand it:
    \begin{itemize}
       \item Detect violation of not wearing belts - \textcolor{blue}{ACHIEVED}
       \item People Count - \textcolor{blue}{ACHIEVED}
       \item Face Detection - \textcolor{blue}{ACHIEVED}
       \item Gait Detection(Mobile Phone Usage) - \textcolor{green}{To Be Achieved}
       \item Desktop Comprehensive Footage Analysis Tool - \textcolor{red}{Work in Progress}
       \item Fatigue Detection - \textcolor{green}{To Be Achieved}
       \item Event Based local recording - \textcolor{yellow}{Partially Achieved} 
    \end{itemize}
    Additional things to be implemented:
    \begin{itemize}
        \item Camera Health monitoring
        \item Device Health
        \item System Status
        \item Cloud API based Analytics
    \end{itemize}
\textbf{SOLUTION:} \\
    Solution currently implemented for onboard event detection using OpenCV's face detection.
    \begin{itemize}
        \item Face recognition may be achieved through an OpenCV based library 
        \href{https://github.com/uricamic/clandmark}{\underline{\textcolor{blue}{clandmark}}}. 
        It seems to be a development over flandmark which is an accepted solution to the problem. 
        \item Gait Detection may be achieved through a combination of eye tracking offered by clandmark
        and temporal correlations. Given an audio input from the cabin we might be able to acheive greater 
        accuracy for such events.
        \item Fatigue detection may again be achieved through a combination of eye tracking, temporal correlations
        and location analysis through an input GPS feed if possible. 
        \item Camera health monitoring might be implemented within the camera's API. I am not too sure about the
        information being extracted from the camera. I do not have enough information at hand about the hardware.
    \end{itemize} 
\textbf{TENSORFLOW:}
    All of the information I have at the moment suggests to me that tensorflow might be the most efficient path to our required 
    solution. Although, I think I might see a possible solution using no Machine Learning since the data set (of about 350 drivers)
    is relatively small and might be done through image analysis and slightly complex parallelisation on a GPU to improve the TTS. 
    With the availability of clandmark, I think this solution might be more feasible than I initially thought.
\newpage
\Large{Table of Libraries to the best of my knowledge}
\\ \vspace{1cm} \\
\begin{tabulary}{0.8\textwidth}{LLL}
    \hline
    Requirement & Pre-requisite(s) & Library & Class(es) & Efficiency \\
    \hline  
    Driver Detection & no Meta Data needed & OpenCV & \href{http://docs.opencv.org/2.4/doc/tutorials/objdetect/cascade_classifier/cascade_classifier.html}{\underline{cascade classifier}} & \\ \hline
    Verify Driver & driver details and previously trained data & & & \\
    \hline
  \end{tabulary}
\end{document}